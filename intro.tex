\begin{abstract}

Internet of Things (IoT) promises to connect billions of objects to Internet. After deploying many stand-alone IoT systems in different domains, the current trend is to develop a unified IoT platform so that objects can be made accessible to applications across organizations and domains. Towards this goal, quite a few proposals have been made to build a unified IoT platform as an overlay on today�s Internet. Such an overlay solution, however, is inadequate to address the important challenges posed by a unified IoT system, especially in terms of mobility, scalability, and communication reliability, due to the inherent inefficiencies of the current Internet. To address this problem, we propose to build a unified IoT platform based on the Information Centric Network (ICN) architecture, which we call ICN-IoT. ICN-IoT leverages the salient features of ICN, and thus provides seamless mobility support, scalability, and efficient content and service delivery. Specifically, we explore using two ICN architectures -- MobilityFirst and NDN --  to support IoT, and refer to them as MF-IoT and NDN-IoT, respectively.   

In this paper, we discuss the detailed design of MF-IoT and NDN-IoT, focusing on their service discovery and pub/sub model. For evaluation purpose, we consider two realistic IoT applications scenarios, a smart building scenario and a smart campus bus scenario, with the former representing mostly stationary IoT devices while the latter representing mobile IoT devices. We have also compared the performance of these two approaches through detailed simulations. 

\end{abstract}

\section{Introduction}
\label{sec:intro}

During the past decade, many stand-alone Internet of Things (IoT) systems have been deployed, in domains including smart homes, smart grids, smart transportation, smart healthcare, etc.  The recent trend is to evolve towards a globally unified IoT platform, in which billions of objects connect to the Internet, available for interactions among themselves, as well as interactions with many different applications across boundaries of administration and domains.  Building a unified IoT platform, however, poses a set of unique challenges on the underlying network and systems. To name a few, it needs to support a large number of networked objects\cite{***},  many of which are mobile. These objects will have  heterogeneous means of connecting to the Internet, often with severe resource constraints. Further, interactions between the applications and objects are often real-time and dynamic, requiring strong security and privacy protections.  To address these challenges, we next identify several key requirements of a unified IoT system. 

Cisco predicts there will be around 50 Billion IoT devices such as sensors, RFID tags, and actuators, on the Internet by 2020~cite{***}. The first step towards a unified IoT platform,  is thus the ability to assign names that are unique,  secure, and persistent against  dynamic attributes that are common in IoT systems, such as device mobility.   Further, the underlying platform needs to scale smoothly with respect to the number of devices and the amount of data generated by these devices.  Another challenge for IoT systems is the resource constraints faced by many IoT devices, including constrained resources in power, computing, storage, bandwidth, etc.  Specifically, power constraints limit how much data the devices can process and communicate,  computing constraints limit the type and amount of processing the devices can perform, storage constraints of the devices limit the amount of data that can be stored on the devices, and  bandwidth constraints limit the amount of communication these devices can have. Finally, a unified IoT platform should be able to provide seamless services  in the presence of device mobility.  
